\chapter{ベクトルに関する定理}

\begin{thm}\label{theorem:1}
  \(\vec{a},\ \vec{b}\) を任意のベクトルとするとき、
  \[\vec{a}+\vec{x}=\vec{b}\]
  を満たすベクトル \(\vec{x}\) がただ一つ存在する。
\end{thm}

\begin{proof}
  \(\vec{x}=(-\vec{a})+\vec{b}\) とおくと、
  \begin{align*}
    \vec{a}+\vec{x} &= \vec{a}+\{(-\vec{a})+\vec{b}\}=\{\vec{a}+(-\vec{a})\}+\vec{b} \\
                    &= \vec{0}+\vec{b}=\vec{b}.
  \end{align*}
  ゆえに、\(\vec{x}\) は存在する。

  次に、\(\vec{a}+\vec{y}=\vec{b}\) となる \(\vec{y}\) があるとすると、
  \begin{align*}
               \vec{a}+\vec{x} &= \vec{a}+\vec{y} \\
    -\vec{a}+(\vec{a}+\vec{x}) &= -\vec{a}+(\vec{a}+\vec{y}) \\
    (-\vec{a}+\vec{a})+\vec{x} &= (-\vec{a}+\vec{a})+\vec{y} \\
               \vec{0}+\vec{x} &= \vec{0}+\vec{y} \\
                       \vec{x} &= \vec{y}.
  \end{align*}
  ゆえに、\(\vec{x}\) はただ一つ。
\end{proof}

\begin{dfn}
  \(\vec{a}+\vec{x}=\vec{b}\) を満たすベクトル \(\vec{x}\) を
  \[\vec{a}-\vec{b}\]
  と表す。
\end{dfn}

\begin{thm}\label{theorem:2}
  \(0\vec{a}=\vec{0}.\)
\end{thm}

\begin{proof}
  まず、
  \[0\vec{a}+0\vec{a}=(0+0)\vec{a}=0\vec{a}.\]
  また、
  \[0\vec{a}+\vec{0}=0\vec{a}.\]
  ゆえに、定理 \ref{theorem:1} より
  \[0\vec{a}=0.\]
\end{proof}

\begin{thm}\label{theorem:3}
  \(\lambda\vec{0}=\vec{0}.\)
\end{thm}

\begin{proof}
  まず、
  \[\lambda\vec{0}+\lambda\vec{0}=\lambda(\vec{0}+\vec{0})=\lambda\vec{0}.\]
  また、
  \[\lambda\vec{0}+\vec{0}=\lambda\vec{0}.\]
  ゆえに、定理 \ref{theorem:1} より
  \[\lambda\vec{0}=\vec{0}.\]
\end{proof}

\begin{thm}\label{theorem:4}
  \((-1)\vec{a}=-\vec{a}.\)
\end{thm}

\begin{proof}
  まず、
  \begin{align*}
    \vec{a}+(-1)\vec{a} &= 1\vec{a}+(-1)\vec{a}=\{1+(-1)\}\vec{a} \\
                        &= 0\vec{a}=\vec{0}.
  \end{align*}
  また、
  \[\vec{a}+(-\vec{a})=\vec{0}.\]
  ゆえに、定理 \ref{theorem:1} より
  \[(-1)\vec{a}=-\vec{a}.\]
\end{proof}

\begin{thm}\label{theorem:5}
  \(\lambda\vec{a}=\vec{0}\) であるための必要十分条件は、\(\lambda=0\) または \(\vec{a}=\vec{0}.\)
\end{thm}

\begin{proof}
  まず、必要条件であることを示す。
  \(\lambda=0\) ならばそれでよし、ゆえに \(\lambda\neq0\) とすると、
  \begin{align*}
               \frac{1}{\lambda}(\lambda\vec{a}) &= \frac{1}{\lambda}\vec{0} \\
    \left(\frac{1}{\lambda}\lambda\right)\vec{a} &= \vec{0} \\
                                        1\vec{a} &= \vec{0} \\
                                         \vec{a} &= \vec{0}.
  \end{align*}
  次に、十分条件であることを示す。
  これは定理 \ref{theorem:2} と \ref{theorem:3} より明らか。
\end{proof}
