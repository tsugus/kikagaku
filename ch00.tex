\chapter*{前書き}

これは初等幾何学を系統的に述べようとするものではありません。
作者の私的なまとめです。

参考文献 \cite{seki} の内容に接続するのが狙いです。
私としては、公理すべてがまとめて示されて、しかもそのために要する定義はなるべく少ないのが理想です。
\cite{seki} は良書ですが、しかし、そのようには書かれていません。
それゆえ、私好みに若干変更された公理系と定義でもって開始しながら、いかにその書物の主要な内容へと再び繋げるか、ということが、私が本文の内容をまとめたときのささやかな試みでした。
したがって、いわばアダプターのようなものなので、学問的に不足のない内容ではないことをご了承ください。

公理系の特徴として、集合、実数、多項式の展開・因数分解については既知としています(これは参考にしたものがもともとそうです)。ゆえに、厳密さを担保したいならば、今まで既知としていた部分を集合論や実数論に丸投げすればよいし、また、わかりやすさを打ち出したいならば、中学数学の説明を援用すればよい。

それゆえ予備知識としては、中学数学で概ね間に合うはずですが、その範囲を逸脱する用語や記号も断りなしに少し使っています。
しかしながら、ちょっとネットで検索すればすぐ理解できるようなことだと個人的には思います。

なお、本文において「ベクトル」は無定義語(定義されないが、公理によって意味が定まるもの)であるから、いきなり出てきても驚く必要はありません。
