\chapter{補遺}

\begin{dfn*}[関数が連続]
  関数 \(f\) が\textbf{連続}\index{れんぞく@(関数が)連続}とは、任意の実数 \(\varepsilon>0\) に対して、実数 \(\delta>0\) が存在し、すべての実数 \(x\) について
  \[|x-a|<\delta \quad \text{ならば} \quad |f(x)-f(a)|<\varepsilon\]
  となることである。
\end{dfn*}

\begin{thm*}[中間値の定理]
  関数 \(f(x)\) が閉区間 \(a\leqq x\leqq b\) で連続で、\(f(a)\neq f(b)\) ならば、\(f(a)\) と \(f(b)\) の間のどのような実数 \(k\) に対しても実数 \(c\) が存在して、
  \[f(c)=k \quad (a<c<b)\]
  となる。
\end{thm*}

\begin{proof}
  \(f(a)<f(b)\) の場合。
  \[A=\{x\in\mathbb{R}\mid a\leqq x\leqq b,\ f(x)<k\}\]
  と定義される実数の集合 \(A\) ついて考える。
  \(x\in A\) ならば \(x\leqq b\) であるから、\(A\) は上に有界で、実数の連続性により上限をもつ。
  よって、
  \[c=\sup A\]
  とおく。

  さて、もし \(f(c)>k\) だとするならば、以下のように矛盾する。
  \(c\) が上限であることより、任意の \(\delta>0\) に対して
  \[c-\delta<x\leqq c,\quad x\in A\]
  となる \(x\) が存在するが、\(x\in A\) より \(f(x)<k.\)
  しかし、\(f\) が連続であることにより、\(\delta>0\) を十分小さくとれば \(|f(x)-f(c)|<f(c)-k\) とできるので、\(f(x)>k\) となって矛盾。

  また、もし \(f(c)<k\) だとするならば、以下のように矛盾する。
  \(f(b)>k\) なので \(b\neq c.\)
  つまり \(c<b.\)
  ゆえに \(f\) の連続性により、\(\delta>0\) を十分小さくとれば、
  \[c<x<c+\delta<b\]
  を満たす \(x\) に対して \(|f(x)-f(c)|<k-f(c)\) とすることができる。
  すなわち \(f(x)<k.\)
  ところが、この \(x\) は \(A\) に属すべきにもかかわらず、\(A\) の上限の \(c\) より大きいので、矛盾。

  以上により、\(f(c)=k.\)

  \(f(a)>f(b)\) の場合も同様である。
\end{proof}
