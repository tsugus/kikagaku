\documentclass[dvipdfmx,12pt]{jsreport}

\usepackage[
  a5paper, % A5判に設定
  top=20mm, % 上余白を20mmに設定
  bottom=20mm, % 下余白を20mmに設定
  left=20mm, % 左余白を20mmに設定
  right=20mm % 右余白を20mmに設定
]{geometry}

\usepackage{makeidx}
\usepackage{amsmath}
\usepackage{amssymb}
\usepackage{amsthm}
\usepackage{bm}

\makeindex
\theoremstyle{definition}
\newtheorem{dfn}{定義}
\newtheorem*{dfn*}{定義}
\theoremstyle{plain}
\newtheorem{axm}{公理}
\newtheorem*{axm*}{公理}
\newtheorem{thm}{定理}
\newtheorem*{thm*}{定理}
\newtheorem{col}{系}
\newtheorem*{col*}{系}
\newtheorem{lmm}{補題}
\newtheorem*{lmm*}{補題}

\renewcommand{\proofname}{\textbf{証明}}

\title{初等幾何学の現代的公理系}
\date{2025.10.26\\02:45}
\author{Tsugu}

\begin{document}
  \maketitle
  \chapter*{前書き}

これは初等幾何学を系統的に述べようとするものではありません。
作者の私的なまとめです。

参考文献 \cite{seki} の内容に接続するのが狙いです。
私としては、公理すべてがまとめて示されて、しかもそのために要する定義はなるべく少ないのが理想です。
\cite{seki} は良書ですが、しかし、そのようには書かれていません。
それゆえ、私好みに若干変更された公理系と定義でもって開始しながら、いかにその書物の主要な内容へと再び繋げるか、ということが、私が本文の内容をまとめたときのささやかな試みでした。
したがって、いわばアダプターのようなものなので、学問的に不足のない内容ではないことをご了承ください。

公理系の特徴として、集合、実数、多項式の展開・因数分解については既知としています(これは参考にしたものがもともとそうです)。ゆえに、厳密さを担保したいならば、今まで既知としていた部分を集合論や実数論に丸投げすればよいし、また、わかりやすさを打ち出したいならば、中学数学の説明を援用すればよい。

それゆえ予備知識としては、中学数学で概ね間に合うはずですが、その範囲を逸脱する用語や記号も断りなしに少し使っています。
しかしながら、ちょっとネットで検索すればすぐ理解できるようなことだと個人的には思います。

なお、本文において「ベクトル」は無定義語(定義されないが、公理によって意味が定まるもの)であるから、いきなり出てきても驚く必要はありません。

  \tableofcontents
  \chapter{初等幾何学の公理系}

\section{定義}

\begin{dfn}[\(\bm{E}\)]\label{definition:1}
  点というものすべての集合を
  \[\bm{E}\]
  と書く。
  (ここでは、\(\bm{E}\) の元を \(A,\ B,\ C,\ \ldots\) のようにラテン文字の大文字で表し、\(E\) の部分集合を断りなくしばしばラテン文字の小文字で表すことにする。)
\end{dfn}

\begin{dfn}[\(\bm{V}\)]\label{definition:2}
  ベクトルというものすべての集合を
  \[\bm{V}\]
  と書く。
  (一般に、\(\bm{V}\) の元を \(\vec{a},\ \vec{b},\ \vec{c},\ \ldots\) のようにラテン文字の小文字の上に右向きの矢印をつけて表す。)
\end{dfn}

\begin{dfn}[スカラー]\label{definition:3}
  実数のことをベクトルと区別して\textbf{スカラー}\index{すからー@スカラー}という。
  (ここでは、スカラーを \(\alpha,\ \beta,\ \gamma,\ \ldots\) のようにギリシャ文字の小文字で表す。)
\end{dfn}

\begin{dfn}[倍]\label{definition:4}
  任意のスカラー \(\lambda\) と任意のベクトル \(\vec{a}\) に対して、それらの積 \(\lambda\vec{a}\)(公理 \ref{axiom:5} によってただ一つ存在することが保証される)を、\(\vec{a}\) の \(\lambda\) \textbf{倍}\index{ばい@倍}ともいう。
\end{dfn}

\begin{dfn}[スカラー倍]\label{definition:5}
  スカラーとベクトルの積を、ベクトルの\textbf{スカラー倍}\index{すからーばい@スカラー倍}という。
\end{dfn}

\begin{dfn}[直線、端点]\label{definition:6}
  異なる任意の二点 \(A,\ B\) に対して、(公理 \ref{axiom:10} の記法の下で)
  \[\overrightarrow{AX}=\lambda\overrightarrow{AB}\]
  となる実数 \(\lambda\) が存在する点 \(X\) すべての集合を、点 \(A,\ B\) を通る\textbf{直線}\index{ちょくせん@直線}といい、
  \[\text{直線} AB\]
   と表す。
\end{dfn}

\begin{dfn}[一直線上にある]\label{definition:7}
  複数の点が同一の直線に含まれうるとき、それらは\textbf{一直線上にある}\index{いっちょくせんじょうにある@一直線上にある}という。
\end{dfn}

\begin{dfn}[半直線、端点]\label{definition:8}
  異なる任意の二点 \(A,\ B\) に対して、
  \[\overrightarrow{AX}=\lambda\overrightarrow{AB},\quad \lambda\geqq0\]
  となる実数 \(\lambda\) が存在する点 \(X\) すべての集合を、点 \(A\) を\textbf{端点}\index{たんてん@端点}とし点 \(B\) を通る\textbf{半直線}\index{はんちょくせん@半直線}といい、
  \[\text{半直線} AB\]
   と表す。
  (この表記法は点の順番によって端点を示すが、絶対的な決まりではない。)
\end{dfn}

\begin{dfn}[平面]\label{definition:9}
  一直線上にない任意の三点 \(A,\ B,\ C\) に対して、
  \[\overrightarrow{AX}=\lambda\overrightarrow{AB}+\mu\overrightarrow{AC}\]
  となる実数 \(\lambda,\ \mu\) が存在する点 \(X\) すべての集合を、点 \(A,\ B,\ C\) が定める\textbf{平面}\index{へいめん@平面}という。ただし、公理 \ref{axiom:14} によって高々一つしかないので普通は単に平面という。
\end{dfn}

\begin{dfn}[半平面、へり]\label{definition:10}
  一直線上にない任意の三点 \(A,\ B,\ C\) に対して、
  \[\overrightarrow{AX}=\lambda\overrightarrow{AB}+\mu\overrightarrow{AC},\quad \mu\geqq0\]
  となる実数 \(\lambda,\ \mu\) が存在する点 \(X\) すべての集合を、直線 \(AB\) を\textbf{へり}\index{へり@へり}とし点 \(C\) を含む\textbf{半平面}\index{はんへいめん@半平面}という。また、
  \[\overrightarrow{AX}=\lambda\overrightarrow{AB}+\mu\overrightarrow{AC},\quad \mu\leqq0\]
  となる実数 \(\lambda,\ \mu\) が存在する点 \(X\) すべての集合を、直線 \(AB\) をへりとし点 \(C\) を含まない半平面という。
  なお、この両方をあえて区別せず、直線 \(AB\) をへりとする半平面とよぶことがある。
\end{dfn}

(注意点として、「点 \(C\) を含む半平面」の「含む」は述語ではなく単に名称の一部だということである。よって、\(C\) がへりの点である場合、実際に含まれる点にもかかわらず正しく指示機能を果たさない。)

\begin{dfn}[零角]\label{definition:11}
  任意の三点 \(O,\ A,\ B\) に対して、半直線 \(OA\) と半直線 \(OB\) とが一致するとき、半直線 \(OA\) および半直線 \(OB\) を\textbf{零角}\index{れいかく@零角}といい、
  \[\text{零角} AOB\]
  と表す。
\end{dfn}

\begin{dfn}[凸角]\label{definition:12}
  任意の三点 \(O,\ A,\ B\) に対して、半直線 \(OA\) と半直線 \(OB\) とが一致もせず合わさって一つの直線もなさないとき、直線 \(OA\) をへりとし点 \(B\) を含む半平面と、直線 \(OB\) をへりとし点 \(A\) を含む半平面との共通部分を\textbf{凸角}\index{とつかく@凸角}といい、
  \[\text{凸角} AOB\]
  と表す。
\end{dfn}

\begin{dfn}[平角]\label{definition:13}
  任意の三点 \(O,\ A,\ B\) に対して、半直線 \(OA\) と半直線 \(OB\) とが合わさって一つの直線をなすとき、直線 \(AB\) をへりとする半平面を\textbf{平角}\index{へいかく@平角}といい、
  \[\text{平角} AOB\]
  と表す。
  ただし、この記法からだけでは、二つある半平面のどちらか分からない。
  それゆえ、厳密さを保つためには、どちらの半平面か別に指定する必要がある。
\end{dfn}

\begin{dfn}[凹角]\label{definition:14}
  任意の三点 \(O,\ A,\ B\) に対して、半直線 \(OA\) と半直線 \(OB\) とが一致もせず合わさって一つの直線もなさないとき、平面から凸角 \(AOB\) を取り去った残りの部分に半直線 \(OA\) と半直線 \(OB\) を合わせた全体を\textbf{凹角}\index{おうかく@凹角}といい、
  \[\text{凹角} AOB\]
  と表す。
\end{dfn}

\begin{dfn}[周角]\label{definition:15}
  任意の三点 \(O,\ A,\ B\) に対して、半直線 \(OA\) と半直線 \(OB\) とが一致するとき、平面から零角 \(AOB\) を取り去った残りの部分に半直線 \(OA\) と半直線 \(OB\) を合わせた全体、すなわち平面全体を\textbf{周角}\index{しゅうかく@周角}といい、
  \[\text{周角} AOB\]
  と表す。
\end{dfn}

\begin{dfn}[角、辺]\label{definition:16}
  零角、凸角、平角、凹角、周角をまとめて\textbf{角}\index{かく@角}という。
  \(O,\ A,\ B\) を任意の三点とするとき、\(\text{角} AOB\) があるならば、それは
  \[\angle AOB\]
  と書かれ、半直線 \(OA\) と半直線 \(OB\) はそれぞれ \(\angle AOB\) の\textbf{辺}\index{へん@(角の)辺}とよばれる。
  二つの辺が半直線 \(h,\ k\) である角を
  \[\angle hk\]
  と表す。
\end{dfn}

\begin{dfn}[ノルム]\label{definition:17}
  任意のベクトル \(\vec{a}\) に対して、\textbf{ノルム}\index{のるむ@ノルム}(norm)とよばれる実数 \(\|\vec{a}\|\) を
  \[\|\vec{a}\|=\sqrt{(\vec{a},\vec{a})}\]
  と定義する。
\end{dfn}

\begin{dfn}[距離]\label{definition:18}
  任意の二点 \(P,Q\) に対して、二点間の\textbf{距離}\index{きょり@(二点間の)距離} \(d(P,Q)\) を
  \[d(P,Q)=\|\overrightarrow{PQ}\|\]
  と定義する。
\end{dfn}

\section{ベクトルの公理群}

\begin{axm}\label{axiom:1}
  任意のベクトル \(\vec{a},\ \vec{b}\) に対して、ベクトルの\textbf{和}\index{わ@和}とよばれるベクトル
  \[\vec{a}+\vec{b}\]
  がただ一つ存在する。
\end{axm}

\begin{axm}\label{axiom:2}
  \textbf{零ベクトル}\index{れいべくとる@零ベクトル}とよばれるベクトル
  \[\vec{0}\]
  がただ一つ存在する。
\end{axm}

\begin{axm}\label{axiom:3}
  任意のベクトル \(\vec{a}\) に対して、その\textbf{逆向きのベクトル}\index{ぎゃくむきのべくとる@逆向きのベクトル}とよばれるベクトル
  \[-\vec{a}\]
  がただ一つ存在する。
\end{axm}

\begin{axm}\label{axiom:4}
  任意のベクトル \(\vec{a},\ \vec{b},\ \vec{c}\) に対して、
  \begin{enumerate}
    \item \(\vec{a}+\vec{b}=\vec{b}+\vec{a}\)
    \item \((\vec{a}+\vec{b})+\vec{c}=\vec{a}+(\vec{b}+\vec{c})\)
    \item \(\vec{a}+\vec{0}=\vec{0}+\vec{a}=\vec{a}\)
    \item \(\vec{a}+(-\vec{a})=(-\vec{a})+\vec{a}=\vec{0}\)
  \end{enumerate}
  が成り立つ。
\end{axm}

\begin{axm}\label{axiom:5}
  任意のスカラー \(\lambda\) と任意のベクトル \(\vec{a}\) に対して、\(\lambda\) と \(\vec{a}\) の\textbf{積}\index{せき@(スカラーとベクトルの)積}とよばれるベクトル
  \[\lambda\vec{a}\]
  がただ一つ存在する。
\end{axm}

\begin{axm}\label{axiom:6}
  任意のスカラー \(\lambda,\ \mu\) と任意のベクトル \(\vec{a},\ \vec{b}\) に対して、
  \begin{enumerate}
    \item \(\lambda(\vec{a}+\vec{b})=\lambda\vec{a}+\lambda\vec{b}\)
    \item \((\lambda+\mu)\vec{a}=\lambda\vec{a}+\mu\vec{a}\)
    \item \(\lambda(\mu\vec{a})=(\lambda\mu)\vec{a}\)
    \item \(1\vec{a}=\vec{a}\)
  \end{enumerate}
  が成り立つ。
\end{axm}

\section{内積の公理群}

\begin{axm}\label{axiom:7}
  任意のベクトル \(\vec{a},\ \vec{b}\) に対して、\(\vec{a}\) と \(\vec{b}\) の\textbf{内積}\index{ないせき@内積}とよばれる実数
  \[(\vec{a},\vec{b})\]
  がただ一つ存在する。
\end{axm}

\begin{axm}\label{axiom:8}
  任意のベクトル \(\vec{a},\ \vec{b},\ \vec{c}\) と任意のスカラー \(\lambda\) に対して、
  \begin{enumerate}
    \item \((\vec{a},\vec{b})=(\vec{b},\vec{a})\)
    \item \((\vec{a}+\vec{b},\vec{c})=(\vec{a},\vec{c})+(\vec{b},\vec{c})\)
    \item \((\lambda\vec{a},\vec{b})=\lambda(\vec{a},\vec{b})\)
    \item \((\vec{a},\vec{a})\geqq0\)
    \item \((\vec{a},\vec{a})=0\) ならば、かつそのときに限り \(\vec{a}=\vec{0}\)
  \end{enumerate}
  が成り立つ。
\end{axm}

\section{ユークリッド空間の公理群}

\begin{axm}\label{axiom:9}
  少なくとも一つの点が存在する。
\end{axm}

\begin{axm}\label{axiom:10}
  任意の二点 \(P,\ Q\) に対して、ただ一つのベクトル
  \[\overrightarrow{PQ}\]
  が存在する。
  \(P\) と \(Q\) は一致してもよい。
\end{axm}

\begin{axm}\label{axiom:11}
  任意の点 \(P\) と任意のベクトル \(\vec{a}\) に対して、
  \[\overrightarrow{PQ}=\vec{a}\]
  を満たす点 \(Q\) がただ一つ存在する。
\end{axm}

\begin{axm}\label{axiom:12}
  任意の三点 \(P,\ Q,\ R\) に対して、
  \[\overrightarrow{PR}+\overrightarrow{RQ}=\overrightarrow{PQ}\]
  が成り立つ。
\end{axm}

\section{次元の公理群}

\begin{axm}\label{axiom:13}
  少なくとも二つのベクトルが存在し、一方は他方のスカラー倍でない。
\end{axm}

\begin{axm}\label{axiom:14}
  任意の三点 \(A,\ B,\ C\) が一直線上にないならば、どのような点 \(X\) に対しても、
  \[\overrightarrow{AX}=\lambda\overrightarrow{AB}+\mu\overrightarrow{AC}\]
  を満たす実数 \(\lambda,\ \mu\) がただ一組存在する。
\end{axm}

\section{角の大きさの公理群}

\begin{axm}\label{axiom:15}
  角の辺を \(h,\ k\) とするとき、角 \(\angle{hk}\) には、\textbf{角の大きさ}\index{かくのおおきさ@角の大きさ}とよばれるただ一つの実数
  \[\overline{\angle{hk}}\]
  が対応する。
\end{axm}

\begin{axm}\label{axiom:16}
  角の辺を \(h,\ k\) とするとき、
  \(\overline{\angle{hk}}=0\) ならば \(h=k\) である。
\end{axm}

\begin{axm}\label{axiom:17}
  \(\angle{hk}\) と \(\angle{kl}\) が辺以外に共通部分を持たず、ともに \(\angle{hl}\) の部分集合であるならば、
  \[\overline{\angle{hl}}=\overline{\angle{hk}}+\overline{\angle{kl}}\]
  が成り立つ。
\end{axm}

\begin{axm}\label{axiom:18}
  二つの任意の角、\(\angle{AOB}\) と \(\angle{A'O'B'}\) に対して、
  \[d(O,A)=d(O',A'),\quad d(O,B)=d(O',B'),\quad d(A,B)=d(A',B')\]
  ならば
  \[\overline{\angle{AOB}}=\overline{\angle{A'O'B'}}\]
  である。
\end{axm}

  \chapter{ベクトルに関する定理}

\begin{thm}\label{theorem:1}
  \(\vec{a},\ \vec{b}\) を任意のベクトルとするとき、
  \[\vec{a}+\vec{x}=\vec{b}\]
  を満たすベクトル \(\vec{x}\) がただ一つ存在する。
\end{thm}

\begin{proof}
  \(\vec{x}=(-\vec{a})+\vec{b}\) とおくと、
  \begin{align*}
    \vec{a}+\vec{x} &= \vec{a}+\{(-\vec{a})+\vec{b}\}=\{\vec{a}+(-\vec{a})\}+\vec{b} \\
                    &= \vec{0}+\vec{b}=\vec{b}.
  \end{align*}
  ゆえに、\(\vec{x}\) は存在する。

  次に、\(\vec{a}+\vec{y}=\vec{b}\) となる \(\vec{y}\) があるとすると、
  \begin{align*}
               \vec{a}+\vec{x} &= \vec{a}+\vec{y} \\
    -\vec{a}+(\vec{a}+\vec{x}) &= -\vec{a}+(\vec{a}+\vec{y}) \\
    (-\vec{a}+\vec{a})+\vec{x} &= (-\vec{a}+\vec{a})+\vec{y} \\
               \vec{0}+\vec{x} &= \vec{0}+\vec{y} \\
                       \vec{x} &= \vec{y}.
  \end{align*}
  ゆえに、\(\vec{x}\) はただ一つ。
\end{proof}

\begin{dfn}
  \(\vec{a}+\vec{x}=\vec{b}\) を満たすベクトル \(\vec{x}\) を
  \[\vec{a}-\vec{b}\]
  と表す。
\end{dfn}

\begin{thm}\label{theorem:2}
  \(0\vec{a}=\vec{0}.\)
\end{thm}

\begin{proof}
  まず、
  \[0\vec{a}+0\vec{a}=(0+0)\vec{a}=0\vec{a}.\]
  また、
  \[0\vec{a}+\vec{0}=0\vec{a}.\]
  ゆえに、定理 \ref{theorem:1} より
  \[0\vec{a}=0.\]
\end{proof}

\begin{thm}\label{theorem:3}
  \(\lambda\vec{0}=\vec{0}.\)
\end{thm}

\begin{proof}
  まず、
  \[\lambda\vec{0}+\lambda\vec{0}=\lambda(\vec{0}+\vec{0})=\lambda\vec{0}.\]
  また、
  \[\lambda\vec{0}+\vec{0}=\lambda\vec{0}.\]
  ゆえに、定理 \ref{theorem:1} より
  \[\lambda\vec{0}=\vec{0}.\]
\end{proof}

\begin{thm}\label{theorem:4}
  \((-1)\vec{a}=-\vec{a}.\)
\end{thm}

\begin{proof}
  まず、
  \begin{align*}
    \vec{a}+(-1)\vec{a} &= 1\vec{a}+(-1)\vec{a}=\{1+(-1)\}\vec{a} \\
                        &= 0\vec{a}=\vec{0}.
  \end{align*}
  また、
  \[\vec{a}+(-\vec{a})=\vec{0}.\]
  ゆえに、定理 \ref{theorem:1} より
  \[(-1)\vec{a}=-\vec{a}.\]
\end{proof}

\begin{thm}\label{theorem:5}
  \(\lambda\vec{a}=\vec{0}\) であるための必要十分条件は、\(\lambda=0\) または \(\vec{a}=\vec{0}.\)
\end{thm}

\begin{proof}
  まず、必要条件であることを示す。
  \(\lambda=0\) ならばそれでよし、ゆえに \(\lambda\neq0\) とすると、
  \begin{align*}
               \frac{1}{\lambda}(\lambda\vec{a}) &= \frac{1}{\lambda}\vec{0} \\
    \left(\frac{1}{\lambda}\lambda\right)\vec{a} &= \vec{0} \\
                                        1\vec{a} &= \vec{0} \\
                                         \vec{a} &= \vec{0}.
  \end{align*}
  次に、十分条件であることを示す。
  これは定理 \ref{theorem:2} と \ref{theorem:3} より明らか。
\end{proof}

  \chapter{内積に関する定理}

\begin{thm}\label{theorem:6}
  \((\vec{a},\vec{b}+\vec{c})=(\vec{a},\vec{b})+(\vec{a},\vec{c})\)
\end{thm}

\begin{proof}
  \begin{align*}
    (\vec{a},\vec{b}+\vec{c}) &= (\vec{b}+\vec{c},\vec{a}) \\
                              &= (\vec{b}.\vec{a})+(\vec{c},\vec{a}) \\
                              &= (\vec{a},\vec{b})+(\vec{a},\vec{c}). \\
  \end{align*}
\end{proof}

\begin{thm}\label{theorem:7}
  \((\vec{a},\lambda\vec{b})=\lambda(\vec{a},\vec{b}).\)
\end{thm}

\begin{proof}
  \[(\vec{a},\lambda\vec{b})=(\lambda\vec{b},\vec{a})=\lambda(\vec{b},\vec{a})=\lambda(\vec{a},\vec{b}).\]
\end{proof}

\begin{thm}\label{theorem:8}
  \((\vec{0},\vec{a})=\vec{0}.\)
\end{thm}

\begin{proof}
  \[(\vec{0},\vec{a})+(\vec{0},\vec{a})=(\vec{0}+\vec{0},\vec{a})=(\vec{0},\vec{a}).\]
  ゆえに
  \[(\vec{0},\vec{a})=(\vec{0},\vec{a})-(\vec{0},\vec{a})=0.\]
\end{proof}

\begin{col*}
  \((\vec{a},\vec{0})=\vec{0}.\)
\end{col*}

\begin{thm}[シュワルツ(Schwarz)の不等式]\label{theorem:9}
  \(|(\vec{a},\vec{b})|\leqq\|\vec{a}\|\|\vec{b}\|.\)
\end{thm}

\begin{proof}
  \(\vec{a}=0\) ならば、定理 \ref{theorem:8} より明らかに成り立つ。

  ゆえに \(\vec{a}\neq0\) とする。公理 \ref{axiom:8} より、任意の実数 \(\lambda\) に対して
  \[(\lambda\vec{a}+\vec{b},\lambda\vec{a}+\vec{b})\geqq0.\]
  公理 \ref{axiom:8} を用いて左辺を展開すると、
  \[(\vec{a},\vec{a})\lambda^2+2(\vec{a},\vec{b})\lambda+(\vec{b},\vec{b})\geqq0.\]
  公理 \ref{axiom:8} より \((\vec{a},\vec{a})>0\) であるから、左辺を平方完成すると
  \[(\vec{a},\vec{a})\left(\lambda+\frac{(\vec{a},\vec{b})}{(\vec{a},\vec{a})}\right)^2-\frac{(\vec{a},\vec{b})^2-(\vec{a},\vec{a})(\vec{b},\vec{b})}{(\vec{a},\vec{a})}\geqq0.\]
  ゆえに、左辺は \(\displaystyle\lambda=-\frac{(\vec{a},\vec{b})}{(\vec{a},\vec{a})}\) のとき最小値をとり、それが \(\geqq0\) でなければならない。よって
  \[(\vec{a},\vec{b})^2\leqq(\vec{a},\vec{a})(\vec{b},\vec{b}).\]
  ゆえに
  \[|(\vec{a},\vec{b})|\leqq\|\vec{a}\|\|\vec{b}\|.\]
\end{proof}

\begin{thm}\label{theorem:10}
  以下のことが成り立つ。
  \begin{enumerate}
    \item \(\|\vec{a}\|\geqq0\)
    \item \(\|\vec{a}\|=0\) ならば、かつそのときに限り \(\vec{a}=\vec{0}\)
    \item \(\|\vec{a}+\vec{b}\|\leqq\|\vec{a}\|+\|\vec{b}\|\)
    \item \(\|\lambda\vec{a}\|=|\lambda|\|\vec{a}\|.\)
  \end{enumerate}
\end{thm}

\begin{proof}
  \begin{enumerate}
    \item ノルムの定義より明らか。
    \item \(\sqrt{x}=0\) ならば、かつそのときに限り \(x=0.\)
    ゆえに、\(x=(\vec{a},\vec{a})\) とおけば、ノルムの定義と公理 \ref{axiom:8} より明らか。
    \item
      \begin{align*}
        &(\|\vec{a}\|+\|\vec{b}\|)^2-\|\vec{a}+\vec{b}\|^2 \\
        &= (\|\vec{a}\|+\|\vec{b}\|)^2-(\vec{a}+\vec{b},\vec{a}+\vec{b}) \\
        &= \|\vec{a}\|^2+2\|\vec{a}\|\|\vec{b}\|+\|\vec{b}\|^2-((\vec{a},\vec{a})+2(\vec{a},\vec{b})+(\vec{b},\vec{b})) \\
        &= 2(\|\vec{a}\|\|\vec{b}\|-(\vec{a},\vec{b}))\geqq2(\|\vec{a}\|\|\vec{b}\|-|(\vec{a},\vec{b})|)\geqq0.
      \end{align*}
      最後の不等号は定理 \ref{theorem:9} による。
    \item
      \begin{align*}
        \|\lambda\vec{a}\| &= \sqrt{(\lambda\vec{a},\lambda\vec{a})}=\sqrt{\lambda^2(\vec{a},\vec{a})} \\
                           &= |\lambda|\sqrt{(\vec{a},\vec{a})}|=|\lambda|\|\vec{a}\|.
      \end{align*}
  \end{enumerate}
\end{proof}

\begin{dfn}[ベクトルの垂直]\label{definition:20}
  \((\vec{a},\vec{b})=0\) であることを、\(\vec{a}\) と \(\vec{b}\) は\textbf{垂直}\index{すいちょく@(ベクトルの)垂直}であるといい、
  \[\vec{a}\perp\vec{b}\]
  と表す。
\end{dfn}

\begin{thm}\label{theorem:11}
  \(\vec{a}\perp\vec{b}\) ならば \(\vec{b}\perp\vec{a}.\)
\end{thm}

\begin{proof}
  垂直の定義より明らか。
\end{proof}

\begin{thm}\label{theorem:12}
  \(\vec{a}\perp\vec{0}.\)
\end{thm}

\begin{proof}
  定理 \ref{theorem:8} の系より明らか。
\end{proof}

\begin{thm}\label{theorem:13}
  \(\vec{a}\perp\vec{b}\) ならば、かつそのときに限り \(\|\vec{a}+\vec{b}\|^2=\|\vec{a}\|^2+\|\vec{b}\|^2.\)
\end{thm}

\begin{proof}
  まず、
  \begin{align*}
    \|\vec{a}+\vec{b}\|^2 &= (\vec{a}+\vec{b},\vec{a}+\vec{b}) \\
                          &= (\vec{a},\vec{a})+2(\vec{a},\vec{b})+(\vec{b},\vec{b}) \\
                          &= \|\vec{a}\|^2+2(\vec{a},\vec{b})+\|\vec{b}\|^2.
  \end{align*}
  これを踏まえれば以下のようになる。

  \(\vec{a}\perp\vec{b}\) とする。
  すなわち \((\vec{a},\vec{b})=0.\)
  ゆえに \(\|\vec{a}+\vec{b}\|^2=\|\vec{a}\|^2+\|\vec{b}\|^2.
  \) この推論が逆向きにも成り立つことは明らか。
\end{proof}

\begin{thm}\label{theorem:14}
  \(\vec{a}\neq\vec{0}\) であるとき、どのようなベクトル \(\vec{b}\) に対しても
  \[\vec{a}\perp\left(\vec{b}-\frac{(\vec{a},\vec{b})}{\|\vec{a}\|^2}\vec{a}\right)\]
  が成り立つ。
\end{thm}

\begin{proof}
  \begin{align*}
    \left(\vec{a},\vec{b}-\frac{(\vec{a},\vec{b})}{\|\vec{a}\|^2}\vec{a}\right) &= (\vec{a},\vec{b})-\left(\vec{a},\frac{(\vec{a},\vec{b})}{\|\vec{a}\|^2}\vec{a}\right) \\
                              &= (\vec{a},\vec{b})-\frac{(\vec{a},\vec{b})}{\|\vec{a}\|^2}(\vec{a},\vec{a}) \\
                              &= (\vec{a},\vec{b})-\frac{(\vec{a},\vec{b})}{\|\vec{a}\|^2}\|\vec{a}\|^2=0.
  \end{align*}
\end{proof}

\begin{col*}
  零ベクトルでない任意のベクトル \(\vec{a}\)が与えられているとき、任意のベクトル \(\vec{b}\) は、\(\vec{a}\) のスカラー倍と \(\vec{a}\) に垂直なベクトルとの和として表される。
\end{col*}

\begin{thm}\label{theorem:15}
  \(\vec{a}\neq\vec{0}\) であるとき、
  \[\vec{a}\perp(\vec{b}-\lambda\vec{a})\]
  ならば
  \[\lambda=\frac{(\vec{a},\vec{b})}{\|\vec{a}\|^2}\]
  である。
\end{thm}

\begin{proof}
  \begin{align*}
    0 &= (\vec{a},\vec{b}-\lambda\vec{a})=(\vec{a},\vec{b})+(\vec{a},(-\lambda)\vec{a})=(\vec{a},\vec{b})-\lambda(\vec{a},\vec{a}) \\
      &= (\vec{a},\vec{b})-\lambda\|\vec{a}\|^2.
  \end{align*}
\end{proof}

\begin{col*}
  定理 \ref{theorem:14} の系における、その表され方はただ一つである。
\end{col*}

\begin{dfn}\label{definition:21}
  \(\vec{a}\neq\vec{0}\) のとき、
  \[\frac{(\vec{a},\vec{b})}{\|\vec{a}\|^2}\vec{a}\]
  を \(\vec{b}\) の \(\vec{a}\) の上への\textbf{正射影}\index{せいしゃえい@正射影}という。
\end{dfn}

  \chapter{ユークリッド空間の公理から導かれる定理}

\begin{thm}\label{theorem:16}
  直線に含まれる異なるどの二点に対しても、それら二点を通る直線はもとのものと一致する。
\end{thm}

\begin{proof}
  異なる二点 \(P,\ Q\) について、
  \[\overrightarrow{AP}=\alpha\overrightarrow{AB},\quad \overrightarrow{AQ}=\beta\overrightarrow{AB} \quad (\alpha\neq\beta)\]
  とおける。

  さて、
  \[\overrightarrow{PX}=\lambda\overrightarrow{PQ}\]
  を満たす X について
  \[\overrightarrow{AX}-\overrightarrow{AP}=\lambda\overrightarrow{AQ}-\lambda\overrightarrow{AP}.\]
  ゆえに
  \begin{align*}
    \overrightarrow{AX} &= \lambda\overrightarrow{AQ}+(1-\lambda)\overrightarrow{AP} \\
                        &= \{(\beta-\alpha)\lambda+\alpha\}\overrightarrow{AB}.
  \end{align*}

  ここで、\(\beta-\alpha\neq0\) であるから、
  \[(\beta-\alpha)\lambda+\alpha\]
  は \(\lambda\) を変数とする一次式で、すべての実数値をとる。
  ゆえに、実数を渡る新たな変数 \(\lambda'\) を用いて、
  \[\overrightarrow{AX}=\lambda'\overrightarrow{AB}\]
  と書き改めることができ、これはもとの直線を表す式と本質的に変わらない。
\end{proof}

\begin{thm}\label{theorem:17}
  半直線の端点と異なるどの点に対しても、端点とその点とで決まる半直線はもとのものと一致する。
\end{thm}

\begin{proof}
  端点と異なる点 \(P\) について、
  \[\overrightarrow{AP}=\alpha\overrightarrow{AB} \quad (\alpha>0)\]
  とおける。

  さて、
  \[\overrightarrow{AX}=\lambda\overrightarrow{AP},\quad \lambda\geqq0\]
  を満たす X について
  \[\overrightarrow{AX}=\alpha\lambda\overrightarrow{AB}.\]

  ここで
  \[\alpha\lambda\]
  は、負でないすべての実数を渡る。
  ゆえに
  \[\overrightarrow{AX}=\lambda'\overrightarrow{AB},\quad \lambda'\geqq0\]
  とおくことができ、これはもとの半直線を表す式と本質的に同じである。
\end{proof}

\begin{thm}\label{theorem:18}
  ある点を含む半平面に含まれるがへりの上にはないどの点に対しても、同一の直線をへりとしその点を含む半平面はもとのものと一致する。
\end{thm}

\begin{proof}
  そのような点 \(P\) は
  \[\overrightarrow{AP}=\alpha\overrightarrow{AB}+\beta\overrightarrow{AC},\quad \beta>0\]
  と表される。

  ゆえに、新しい半平面は \(\mu\geqq0\) として、ある \(\lambda,\ \mu\) について
  \begin{align*}
    \overrightarrow{AX} &= \lambda\overrightarrow{AB}+\mu\overrightarrow{AP} \\
                        &= \lambda\overrightarrow{AB}+\alpha\mu\overrightarrow{AB}+\beta\mu\overrightarrow{AC} \\
                        &= (\lambda+\alpha\mu)\overrightarrow{AB}+\beta\mu\overrightarrow{AC}
  \end{align*}
  となる点 \(X\) の集合として定義される。

  ここで、\(\lambda\) と \(\mu\) は独立だから、\(\mu\) が動かない場合のみを考えても、\(\lambda+\alpha\mu\) はすべての実数を渡る。
  \(\beta\mu\) が非負の実数すべてを渡るのは明らか。
  よって、
  \[\overrightarrow{AX}=\lambda'\overrightarrow{AB}+\mu'\overrightarrow{AC},\quad \mu'\geqq0\]
  とおくことができるが、これはもとの半平面を表す式と実質同じである。
\end{proof}

\begin{thm}\label{theorem:19}
  ある点を含まない半平面に含まれないどの点に対しても、同一の直線をへりとしその点を含まない半平面はもとのものと一致する。
\end{thm}

\begin{proof}
  定理 \ref{theorem:18} の証明における \(\mu\) と \(\mu'\) の正負が反転するだけである。
\end{proof}

\begin{thm}\label{theorem:20}
  異なる三点 \(O,\ A,\ B\) が一直線上にあるとき、半直線 \(OA\) と半直線 \(OB\) は一致するか、合わせると直線 \(AB\) になるかのどちらかである。
\end{thm}

\begin{proof}
  点 \(B\) は半直線 \(OA\) 上にあるかないかのどちらかである。

  前者ならば、定理 \ref{theorem:17} により半直線 \(OA,\ OB\) は一致する。

  後者ならば、一直線上にあるのだから
  \[\overrightarrow{OB}=\alpha\overrightarrow{OA},\quad \alpha<0\]
  である。
  すると半直線 \(OB\) は、\(\lambda\geqq0,\) \(\lambda'\leqq0\) として
  \[\overrightarrow{OX}=\lambda\overrightarrow{OB}=\alpha\lambda\overrightarrow{OA}=\lambda'\overrightarrow{OA}\]
  を満たす \(X\) の集合ということになるので、これが半直線 \(OA\) と合わされば一つの直線と等しくなるのは明らかである。
\end{proof}

  \chapter{次元の公理から導かれる定理}

\begin{thm}\label{theorem:21}
  一直線上にない三点が存在する。
\end{thm}

\begin{proof}
  公理 \ref{axiom:9}, \ref{axiom:13} に公理 \ref{axiom:11} を適用すれば明らか。
\end{proof}

\begin{thm}\label{theorem:22}
  平面はただ一つ存在し、すべての点はその平面に含まれる。
\end{thm}

\begin{proof}
  定理 \ref{theorem:21} と公理 \ref{axiom:14} による。
\end{proof}

\begin{dfn}[交わる、交点]\label{definition:22}
  点の集合が点を共有するとき、それらは\textbf{交わる}\index{まじわる@交わる}といい、共有される点を\textbf{交点}\index{こうてん@交点}という。
\end{dfn}

\begin{thm}\label{theorem:23}
  二直線は、点を共有しないか、一点のみで交わるか、全く一致するかのいずれかである。
\end{thm}

\begin{proof}
  一つより多くの点を共有している場合は、定理 \ref{theorem:16} よりそれら二直線は一致する。
\end{proof}

\begin{dfn}[ベクトルの平行]\label{definition:23}
  零ベクトルでない二つのベクトル \(\vec{a},\ \vec{b}\)は、一方が地方のスカラー倍であるとき、\textbf{平行}\index{へいこう@(ベクトルの)平行}であるといわれ、
  \[\vec{a}/\!/\vec{b}\]
  と書かれる。
\end{dfn}

\begin{thm}\label{theorem:24}
  \(\vec{a}/\!/\vec{b}\) ならば \(\vec{b}/\!/\vec{a}.\)
\end{thm}

\begin{proof}
  \(\vec{a}=\lambda\vec{b}\) と表されたとする。
  \(\vec{a}\neq\vec{0}\) より、\(\lambda\neq0\) でなければならない。
  ゆえに \(\displaystyle \vec{b}=\frac{1}{\lambda}\vec{a}.\)
\end{proof}

\begin{thm}\label{theorem:25}
  \(\vec{a}/\!/\vec{b}\) ならば \(\vec{a}\perp\vec{b}\) でない。
\end{thm}

\begin{proof}
  \(\vec{a}=\lambda\vec{b}\neq\vec{0}\) と表されたとする。
  \[(\vec{a},\vec{b})=(\vec{a},\lambda\vec{a})=\lambda(\vec{a},\vec{a})\neq0.\]
\end{proof}

\begin{thm}\label{theorem:26}
 \(\vec{a},\ \vec{b}\) が零ベクトルでなく、互いに平行でもないとき、\(\lambda\vec{a}+\mu\vec{b}=\vec{0}\) ならば \(\lambda=\mu=0.\)
\end{thm}

\begin{proof}
  \(\overrightarrow{OA}=\vec{a},\) \(\overrightarrow{OB}=\vec{b}\) となるように点をとると、\(\vec{a},\ \vec{b}\) は零ベクトルでなく、互いに平行でもないのだから、\(O,\ A,\ B\) は異なる三点である。

  それゆえ公理 \ref{axiom:14} により、\(\lambda\vec{a}+\mu\vec{b}=\vec{0}\) となるのは \(\lambda=\mu=0\) の場合のみ。
\end{proof}

\begin{dfn}[直線の垂直]\label{definition:24}
  互いに垂直な二つのベクトルをもとに定義しうる二つの直線 \(l,\ m\) は、\textbf{垂直}\index{すいちょく@(直線の)垂直}であるといわれ、
  \[l\perp m\]
  と書かれる。
\end{dfn}

\begin{dfn}[直線の平行]\label{definition:25}
  互いに平行な二つのベクトルをもとに定義しうる二つの直線 \(l,\ m\) は、\textbf{平行}\index{へいこう@(直線の)平行}であるといわれ、
  \[l/\!/m\]
  と書かれる。
\end{dfn}

\begin{thm}\label{theorem:27}
  与えられた直線上の与えられた点を通ってその直線に垂直な直線は、ただ一つ存在する。
\end{thm}

\begin{proof}
  まず、与えられた直線 \(l\) を定めるベクトル \(\vec{a}\) に対し、それと垂直かつ零ベクトルでないベクトルが存在することを示す。
  
  定理 \ref{theorem:14} より、任意のベクトル \(\vec{b}\) に対し、その \(\vec{a}\ (\neq\vec{0})\) に対する正射影のスカラーを \(\lambda\) とすると、
  \[\vec{a}\perp(\vec{b}-\lambda\vec{a}).\]

  さて、\(\vec{b}-\lambda\vec{a}=\vec{0}\) ならば、\(\vec{b}\) は \(\vec{a}\) のスカラー倍である。ゆえに、\(\vec{a}\) のスカラー倍にならないベクトルを \(\vec{b}\) として選べば、\(\vec{b}-\lambda\vec{a}\neq\vec{0}\) である。そして、そのように選ぶことは公理 \ref{axiom:13} により可能である。

  それゆえ、与えられた直線上の点から \(\vec{b}-\lambda\vec{a}\) を用いて新たな点をとれば、垂直な直線が得られる。次は、それがただ一つしかないことを示す。

  点 \(P\) が直線 \(l\) 上にあるとき、\(l\) をへりとする同一の半平面上に二点 \(A,\ B\) があって、それらは \(l\) 上になく、直線 \(PA,\ PB\) は \(l\) に垂直だとする。
  さらに、点 \(Q\) を新たにとって、\(l\) の部分である半直線 \(PQ\) を考える。
  すると、\(A,\ P,\ Q\) は一直線上にないから公理 \ref{axiom:14} を用いて、
  \begin{align*}
    0 &= (\overrightarrow{PQ},\overrightarrow{PB})=(\overrightarrow{PQ},\lambda\overrightarrow{PQ}+\mu\overrightarrow{PA}) \\
      &= \lambda(\overrightarrow{PQ},\overrightarrow{PQ})+\mu(\overrightarrow{PQ},\overrightarrow{PA}) \\
      &= \lambda\|\overrightarrow{PQ}\|^2+\mu(\overrightarrow{PQ},\overrightarrow{PA})=\lambda\|\overrightarrow{PQ}\|^2
  \end{align*}
  ここで、\(P\neq Q\) より \(\|\overrightarrow{PQ}\|\neq0.\)
  ゆえに \(\lambda=0.\)
  よって
  \[\overrightarrow{PB}=\lambda\overrightarrow{PQ}+\mu\overrightarrow{PA}=\mu\overrightarrow{PA}.\]
  すなわち、直線 \(PB\) は直線 \(PA\) と重なる。
\end{proof}

\begin{thm}\label{theorem:28}
  異なる二つの直線が交わるならば、それらは平行でない。
\end{thm}

\begin{proof}
  直線 \(AB,\ CD\) があるとする。
  これらが半直線 \(AB,\ CD\) の端点以外の部分で交わるとしても一般性は失われない。
  その交点を \(O\) とする。

  さて、直線 \(AB\) と直線 \(CD\) がもし平行だとすると、直線 \(CD\) も点 \(C\) を基点にしさえすればベクトル \(\overrightarrow{AB}\) を用いて再定義できるから、
  \[\overrightarrow{AO}=\lambda_1\overrightarrow{AB},\quad \overrightarrow{CO}=\lambda_2\overrightarrow{AB}.\]
  ゆえに
  \[\overrightarrow{AC}=\overrightarrow{OC}-\overrightarrow{OA}=(\lambda_1-\lambda_2)\overrightarrow{AB}.\]
  ところがこれは、\(\lambda_1=\lambda_2\) で \(A\) と \(C\) は一致するか、さもなければ \(C\) が直線 \(AB\) 上にあることになる。
  しかし、いずれも異なる二直線という前提に反する。

  ゆえに、平行ではない。
\end{proof}

\begin{thm}\label{theorem:29}
  異なる二つの直線が平行でないならば、それらは交わる。
\end{thm}

\begin{proof}
  二直線 \(AB,\ CD\) が平行でないとし、\(\overrightarrow{AE}=\overrightarrow{CD}\) となる点 \(E\) をとる。
  このとき、公理 \ref{axiom:14} より \(\overrightarrow{AE}=\lambda\overrightarrow{AB}+\mu\overrightarrow{AC}\) となる \(\lambda,\ \mu\) がある。

  さて、ここで \(\mu=0\) とすると、直線 \(AE\) は直線 \(AB\) と平行であることになり、直線 \(CD\) も 直線 \(AB\) と平行になってしまうので、\(\mu\neq0.\)

  ゆえに、\(\displaystyle \overrightarrow{AF}=-\frac{\lambda}{\mu}\overrightarrow{AB}\) とおくことができ、
  \begin{align*}
    \overrightarrow{CF} &= \overrightarrow{AF}-\overrightarrow{AC}=-\frac{\lambda}{\mu}\overrightarrow{AB}-\overrightarrow{AC} \\
                        &= -\frac{1}{\mu}(\lambda\overrightarrow{AB}+\mu\overrightarrow{AC}) \\
                        &= -\frac{1}{\mu}\overrightarrow{AE}=-\frac{1}{\mu}\overrightarrow{CD}.
  \end{align*}
  これは、直線 \(AB\) 上の点として定義した点 \(F\) が直線 \(CD\) 上にもあることを示している。
\end{proof}

\begin{thm}\label{theorem:30}
  与えられた直線外の点を通ってその直線に平行な直線は、ただ一つ存在する。
\end{thm}

\begin{proof}
  直線および平行の定義から明らか。
\end{proof}

  \chapter{角の大きさの公理から導かれる定理}

\begin{thm}\label{theorem:31}
  零角の大きさは \(0\) である。
\end{thm}

\begin{proof}
  どのような \(\angle{hh}\) に対しても適当に \(\angle{hk}\) を選べば、
  \[\overline{\angle{hh}}+\overline{\angle{hk}}=\overline{\angle{hk}}.\]
  ゆえに
  \[\overline{\angle{hh}}=\overline{\angle{hk}}-\overline{\angle{hk}}=0.\]
\end{proof}

\begin{thm}\label{theorem:32}
  周角の大きさはみな等しい。
\end{thm}

\begin{proof}
  なぜならば周角は、辺はいろいろな場合があれど、実体は平面なのでただ一つだからである。
\end{proof}

\begin{dfn}[ベクトルのなす角]\label{definition:26}
  零ベクトルでない二つのベクトルについて、それらと与えられた任意の点 \(O\) をもとに作られる、端点 \(O\) の半直線を二つの辺とする零角または凸角または平角を、点 \(O\) に関する\textbf{ベクトルのなす角}\index{べくとるのなすかく@ベクトルのなす角}という。
  点を具体的に指定しなくても差し支えがない場合、点 \(O\) は省略される。
\end{dfn}

\begin{thm}\label{theorem:33}
  点に関するベクトルのなす角は、点によらずどれもその大きさが一定である。
\end{thm}

\begin{proof}
  公理 \ref{axiom:18} より明らか。
\end{proof}

\begin{dfn}[直角]\label{definition:27}
  互いに垂直な二つのベクトルのなす角を\textbf{直角}\index{ちょっかく@直角}という。
\end{dfn}

\begin{thm}\label{theorem:34}
  直角の大きさはみな等しく、周角のそれの四分の一である。
\end{thm}

\begin{proof}
  \(\vec{a}\perp\vec{b}\) ならば、定理 \ref{theorem:13} より
  \begin{align*}
    \|\vec{b}-\vec{a}\| &= \|\vec{b}+(-\vec{a})\| \\
                        &= \|\vec{b}\|^2+\|-\vec{a}\|^2= \|\vec{b}\|^2+\|\vec{a}\|^2.
  \end{align*}
  ゆえに、直角にいわば対するベクトルのノルムは、直角を挟むベクトルの二つのノルムだけで決まる。

  つまり、どの直角においても、それを挟むベクトルのノルムを適切に調節して長さをそろえれば、直角に対するベクトルはどれもノルムが等しいことが分かる。
  ゆえに、公理 \ref{axiom:18} より、どの直角も同じ大きさである。

  そして、垂直に交わる二直線によってできる四つの直角を考えれば、定理の後半が成り立つことが分かる。
\end{proof}

\begin{thm}\label{theorem:35}
  平角の大きさはみな等しく、周角のそれの半分である。
\end{thm}

\begin{proof}
  平角の辺に対して垂直な半直線を立てれば、明らかに平角は二つの直角を合わせたものである。
\end{proof}

\begin{dfn}[鋭角、鈍角]\label{definition:28}
  直角より小さい角を\textbf{鋭角}\index{えいかく@鋭角}という。
  直角より大きい角を\textbf{鈍角}\index{どんかく@鈍角}という。
  ただし、平角より大きい角は含まない。
\end{dfn}

\begin{lmm}\label{lemma:1}
 \(\vec{a},\ \vec{b}\) が零ベクトルでなく、互いに平行でもないとき、\(\|\vec{a}\|=\|\vec{b}\|\) ならば、\(\vec{a}+\vec{b}\) と任意の点 \(O\) から作られる半直線は、\(O\) に関する \(\vec{x}\) と \(\vec{y}\) のなす角を二等分する。
\end{lmm}

\begin{proof}
  定理 \ref{theorem:26} より \(\vec{a}+\vec{b}\neq\vec{0}.\)

  ここで \(\overrightarrow{OA}=\vec{a},\) \(\overrightarrow{OB}=\vec{b},\) \(\overrightarrow{OC}=\vec{a}+\vec{b}\) とおくと、点 \(C\) は 直線 \(OA\) をへりとし点 \(B\) を含む半平面にあり、直線 \(OB\) をへりとし点 \(A\) を含む半平面にもある。
  ゆえに、点 \(C\) は 凸角 \(AOB\) に含まれ、\(\angle{AOC},\ \angle{BOC}\) は \(\angle{AOB}\) の部分である。

  そして、
  \[\overrightarrow{AC}=\vec{b},\quad \overrightarrow{BC}=\vec{b},\quad \|\vec{a}\|=\|\vec{b}\|.\]

  ゆえに、公理 \ref{axiom:18} より \(\angle{AOC}=\angle{BOC}.\)
  すなわち、半直線 \(OC\) は \(\angle{AOB}\) を二等分する。
\end{proof}

\begin{lmm}\label{lemma:2}
 \(\vec{a},\ \vec{b}\) が零ベクトルでなく、互いに平行でもないとき、
 \[\vec{a}+\frac{\|\vec{a}\|}{\|\vec{a}\|+\|\vec{b}\|}(\vec{b}-\vec{a})\]
 と任意の点 \(O\) から作られる半直線は、\(O\) に関する \(\vec{a}\) と \(\vec{b}\) のなす角を二等分する。
\end{lmm}

\begin{proof}
  \[\vec{c}=\vec{a}+\frac{\|\vec{a}\|}{\|\vec{a}\|+\|\vec{b}\|}(\vec{b}-\vec{a})\]
  とおくと、
  \[\vec{c}=\frac{\|\vec{b}\|}{\|\vec{a}\|+\|\vec{b}\|}\left(\vec{a}+\|\vec{a}\|\cdot\frac{\vec{b}}{\|\vec{b}\|}\right).\]
  ここで \(\displaystyle \frac{\vec{b}}{\|\vec{b}\|}\) のノルムは \(1.\)
  ゆえに、\(\vec{c}\) に補題 \ref{lemma:1} が適用できて、当補題が成り立つ。
\end{proof}

\begin{thm}\label{theorem:36}
 周角の大きさより小さい正の実数には、それを自らの大きさとする角が必ず存在する。
\end{thm}

\begin{proof}
  まず、鋭角について成り立つことを示す。
  零ベクトルでない互いに垂直なベクトルを \(\vec{a},\ \vec{b}\) とし、\(\vec{a}\) と
  \[\vec{a}+\lambda(\vec{b}-\vec{a}) \quad (0\leqq\lambda\leqq1)\]
  のなす角について考える。

  さて、もし \(\vec{a}+\lambda(\vec{b}-\vec{a})=\vec{0}\) だとすると \((1-\lambda)\vec{a}+\lambda\vec{b}=\vec{0}.\) しかし、定理 \ref{theorem:26} より \(1-\lambda=0\) かつ \(\lambda=0\) となって矛盾。

  したがって、どのような \(\lambda\) にもそれに対応する角がある。
  さらに、\(\lambda=0\) には零角が、\(\lambda=1\) には直角が対応する。
  つまり、実数から角の大きさへの関数 \(f\) が存在する。

  ここで \(\delta_0>0\) とし、
  \begin{gather*}
    \vec{a}+(\lambda+\delta_0)(\vec{b}-\vec{a}) \quad \text{と} \quad \vec{a}+\lambda(\vec{b}-\vec{a}), \\
    \vec{a}+\lambda(\vec{b}-\vec{a}) \quad \text{と} \quad \vec{a}+(\lambda-\delta_0)(\vec{b}-\vec{a})
  \end{gather*}
  のなす角それぞれを、補題 \ref{lemma:2} を用いて繰り返し二等分していくことを考える。
  \[\delta_0>\delta_1>\delta_2>\cdots>\delta_n>\cdots>0\]
  なる値の系列が存在して、公理 \ref{axiom:16}, \ref{axiom:17} より、
  \begin{align*}
    f(\lambda+\delta_n)-f(\lambda) &= \frac{f(\lambda+\delta_0)-f(\lambda)}{2^n}>0, \\
    f(\lambda)-f(\lambda-\delta_n) &= \frac{f(\lambda)-f(\lambda-\delta_0)}{2^n}>0
  \end{align*}
  となる。
  同時に公理 \ref{axiom:16}, \ref{axiom:17} より、
  \[\lambda-\delta_n<x<\lambda+\delta_n\]
  ならば
  \[f(\lambda-\delta_n)<f(x)<f(\lambda+\delta_n)\]
  である。

  ゆえに、任意の \(\varepsilon>0\) が与えられたとき、十分小さな \(\delta>0\) を選んで、
  \begin{align*}
    |x-\lambda|<\delta \quad & \text{ならば} \quad |f(x)-f(\lambda)|<\varepsilon, \\
          0<x-0<\delta \quad & \text{ならば} \quad |f(x)-f(0)|<\varepsilon, \\
          0<1-x<\delta \quad & \text{ならば} \quad |f(x)-f(1)|<\varepsilon
  \end{align*}
  となるようにできる。
  すなわち、関数 \(f(\lambda)\) は閉区間 \(0\leqq\lambda\leqq1\) で連続である。
  なおかつ、\(f(0)\neq f(1).\)
  よって、実数論における中間値の定理より、\(f(\lambda)\) は \(0\) から直角の大きさまでの間のあらゆる実数値をとる。

  直角より大きい角については、直角いくつかと鋭角一つとに分割すれば、明らかであろう。
\end{proof}

\begin{dfn}\label{defintion:29}
  \(\bm{E}\) から \(\bm{E}\) への全単射で、点と点の距離を変えない写像を、等長変換という。
\end{dfn}

\begin{thm}\label{theorem:37}
 等長変換によって移された角は、もとの角と大きさが等しい。
\end{thm}

\begin{proof}
  等長変換によってすべての角が角へと移されるのかは別途証明する必要がある。
  しかし、ともかくも角に移されたのであれば、その大きさが変わらないことは公理 \ref{axiom:18} によって保証される。
\end{proof}

  \chapter{補遺}

\begin{dfn*}[関数が連続]
  関数 \(f\) が\textbf{連続}\index{れんぞく@(関数が)連続}とは、任意の実数 \(\varepsilon>0\) に対して、実数 \(\delta>0\) が存在し、すべての実数 \(x\) について
  \[|x-a|<\delta \quad \text{ならば} \quad |f(x)-f(a)|<\varepsilon\]
  となることである。
\end{dfn*}

\begin{thm*}[中間値の定理]
  関数 \(f(x)\) が閉区間 \(a\leqq x\leqq b\) で連続で、\(f(a)\neq f(b)\) ならば、\(f(a)\) と \(f(b)\) の間のどのような実数 \(k\) に対しても実数 \(c\) が存在して、
  \[f(c)=k \quad (a<c<b)\]
  となる。
\end{thm*}

\begin{proof}
  \(f(a)<f(b)\) の場合。
  \[A=\{x\in\mathbb{R}\mid a\leqq x\leqq b,\ f(x)<k\}\]
  と定義される実数の集合 \(A\) ついて考える。
  \(x\in A\) ならば \(x\leqq b\) であるから、\(A\) は上に有界で、実数の連続性により上限をもつ。
  よって、
  \[c=\sup A\]
  とおく。

  さて、もし \(f(c)>k\) だとするならば、以下のように矛盾する。
  \(c\) が上限であることより、任意の \(\delta>0\) に対して
  \[c-\delta<x\leqq c,\quad x\in A\]
  となる \(x\) が存在するが、\(x\in A\) より \(f(x)<k.\)
  しかし、\(f\) が連続であることにより、\(\delta>0\) を十分小さくとれば \(|f(x)-f(c)|<f(c)-k\) とできるので、\(f(x)>k\) となって矛盾。

  また、もし \(f(c)<k\) だとするならば、以下のように矛盾する。
  \(f(b)>k\) なので \(b\neq c.\)
  つまり \(c<b.\)
  ゆえに \(f\) の連続性により、\(\delta>0\) を十分小さくとれば、
  \[c<x<c+\delta<b\]
  を満たす \(x\) に対して \(|f(x)-f(c)|<k-f(c)\) とすることができる。
  すなわち \(f(x)<k.\)
  ところが、この \(x\) は \(A\) に属すべきにもかかわらず、\(A\) の上限の \(c\) より大きいので、矛盾。

  以上により、\(f(c)=k.\)

  \(f(a)>f(b)\) の場合も同様である。
\end{proof}

  \begin{thebibliography}{99}
    \bibitem{seki} 赤攝也.『現代の初等幾何学』(ちくま学芸文庫). 筑摩書房, 2019.
    \bibitem{tajima} 田島一郎.『イプシロン−デルタ』(数学ワンポイント双書). 共立出版株式会社, 2018.
  \end{thebibliography}
  \clearpage
  \printindex
\end{document}
